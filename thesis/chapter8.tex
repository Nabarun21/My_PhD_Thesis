
%
% Chapter 7
%

\chapter{Results}
\label{results}
In this chapter the results of both the searches are presented. The results for the $\hmue$ search are first presented. Results for the $\Hmue$ search follow.

\subsection{$\hmue$ results}
The resulting distributions of the signal variable (after applying all selection requirements as outlined in ~\ref{evt_sel}) are fit using a binned maximum likelihood fit. The entire procedure is described in detail in ~\ref{stat_meth}. All systematic uncertainties are included as nuisance parameters, and the fit is performed simultaneously across all categories. The BDT response distributions of signal and background are shown superimposed for each category in Fig~\ref{fig:BDT_dist_hmue}. The distribution of $\mcol$ for the $\mcol$-fit analysis are also shown in Fig~\ref{fig:mcol_dist_hmue}. We do not observe an excess of signal over expected background. Hence, upper exclusion limits on $\mathcal{B}(\hmue)$ are set, following the procedure described in ~\ref{exc_cal}. In table ~\ref{table:hmue_limits}, the median expected limits, observed limits and the best fit branching fractions for $\mathcal{B}(\hmue)$ are summarized. As noted earlier in this thesis, the tau lepton coming from the Higgs can also decay hadronically. This channel of the LFV Higgs decay, i.e. $\hmuhad$ is studied in a sister analyses by the same group~\cite{HIG-17-001}. The limits (on $\mathcal{B}(\hmuhad)$) from that search are combined with limits on $\mathcal{B}(\hmue)$, as calculated above. All limts are summarized graphically in Fig.~\ref{fig:hmue_limits_brazil}. The combined best fit branching fraction of  $\mathcal{B}(\hmu)$ is found to be $0.00 \pm 0.12$ for the BDT-fit analysis.


The constraints on $\mathcal{B}(\hmu)$ can be transformed into constraints on Lepton Flavor Violating Yukawa Couplings ($Y_{\Pgm\Pgt},Y_{\Pgt\Pgm}$). These couplings represent the strength of an interaction and are related to the decay width $\Gamma(\hmu)$ in the following way~\ref{Harnik:2012pb}:
\begin{equation}                                                                                                                                                                                                 
\Gamma(\hmu)=\frac{m_{h}}{8\pi}(|Y_{\Pgm\Pgt}|^2 + |Y_{\Pgt\Pgm}|^2).                                                          
\label{eq:yuk1}
\end{equation}

The decay width is also related to the branching fraction, $\mathcal{B}(\hmu)$ according to the following equation:
\begin{equation*}                                                                                                                                                                                                \mathcal{B}(\hmu)=\frac{\Gamma(\hmu)}{\Gamma(\hmu) + \Gamma_{SM}}.
\label{eq:yuk2}
\end{equation*}

,where the SM Higgs decay width is assumed to be $\Gamma_{SM}=4.1$ MeV~\cite{Denner:2011mq} for $m_{\PH}=125$\GeV. Using equations~\ref{eq:yuk1} and ~\ref{eq:yuk2}, we derive the constraints on Yukawa couplings at 95\% CL. The limits and best fit values for the Yukawa couplings are summarized in Table~\ref{table:yuk_coup}. Fig.~\ref{fig:yuk_coup} pictorially summarizes all existing limits on Yukawa couplings from different direct and indirect searches. The limits derived from this search are most stringent till date.

\subsection{$\Hmue$ results}


% % uncomment the following lines,
% if using chapter-wise bibliography
%
% \bibliographystyle{ndnatbib}
% \bibliography{example}
