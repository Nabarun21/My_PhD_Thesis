
%
% Chapter 2
%

\chapter{Theoretical bases}
\label{chap:theory}
In this chapter we describe the theoretical motivations that drive the searches described in this thesis. We start with a description  the standard model (SM), its particle content and interactions and the Higgs mechanism. We then talk about the inadequacies of the SM, and the existence of physics beyond the standard model (BSM). We then outline a few BSM models and how they point towards the possibility of the decays that we search for in this thesis.  

\section{The Standard Model }
\label{sec:SM}
The SM is the result of human endeavors over centuries to understand what we and the world around us are made of, and capture those ideas in beautiful mathematical form. Our understanding of the world around us has refined progressively from the ancient times, when best tools of observation we had were nothing but our own eyes to the current day when we are able we collide particles that make up matter at unprecedented speeds, and have sophisticated tools like the CMS detector to aid us. From the ancient greeks who pondered over philosophical questions about what the basic elements of nature were, to the discovery of electron in 1898 by J.J.Thompson, to Rutherford's famous gold foil experiment, to the discovery of the neutron by James Chadwick in 1932 have been stepping stones towards our understanding of nature and the formulation of SM. During the course of its formulation and after, the SM has accurately explained phenomena alreafy known and  predicted the existence of particles that were discovered later. The last of these particles is the Higgs Boson, discovered in 2012 at CERN by the CMS and the ATLAS experiments~\cite{Aad:2012tfa, Chatrchyan:2012ufa, Chatrchyan:2013lba}. The SM is a gauge theory, in which three of the four known natural forces (strong, electromagnetic, weak and not gravity~\ref{tab:forces}) are represented by the SU(3)xSU(2)xU(1) symmetry group. This symmetry group describes under which transformations the SM is invariant. By Noether's theorem each of the above symmetries associated with the SM Lagrangian is associated with a conserved quantity: color charge, weak isospin and electric charge. The following describes the elementary particles of the SM, the interactions among these and finally, the spontaneous symmetry breaking mechanism.

\subsection{Elementary particles}
There are two kinds of elementary particles in the SM. They are characterized by the intrinsic angular momentum that they carry, i.e. by their spin. Fermions, which has half-integer spins, form the building blocks of matter. Bosons, which have integer spins, are the force-carriers or mediators of interactions.

Fermions are fundamental particles, i.e. they cannot be broken down into further consituents. The space-time evolution of the fermions is described by the Dirac equation and their behavior follows Fermi-Dirac statistics. All fermions are subject to the Pauli exclusion principle. The fermions can be further categorized into two classes depending on their interaction with the strong force. Fermions wich do not interact with the strong force are called leptons, and do not carry any color charge. Quarks carry color charge and interact via the strong force. Both leptons and quarks are further classified into three generations. Each lepton generation consists of a lepton and a neutrino while each quark generation consists of a up type and a down type quark. These are outlined in detail below.

Leptons comprise of the familiar electron (e), its heavier cousins muon ($\Pgm$) and tau lepton ($\Pgt$) which carry the same negative electric charge as the electron ($1.6\times10^{-19} C$).  The heavier leptons $\Pgt$ ($\sim 1.8\,\mathrm{GeV}/c^2$ ) and $\Pgm$ ($\sim 105.7\,\mathrm{MeV}/c^2$) have short lifetimes of $\sim 2.9\times 10^{-13}\,$s and $\sim 2.2\times 10^{-6}\,$s respectively. The eventually decay into an electron which is the lightest lepton ($\sim 0.5\,\mathrm{MeV}/c^2$ ) and has infinite lifetime, or lighter hadrons. In the CMS detector, the $\mu$ survives long enough to reach the muon systems is thus detected as its own distinct signature. The $\Pgt$ on the other hand, owing to its extremely short lifetime, can travel only a very short distance ($\sim <10\,mm$) before decaying. Thus, only decay products of tau leptons are able to be directly detected by CMS. Each charged lepton is associated with an electrically neutral neutrino. They are called electron neutrino ($\nu_e$), muon neutrino ($\nu_{\mu}$) and tau neutrino ($\nu_{\mu}$). Because neutrinos carry no electric charge, they do not interact via electromagnetic interaction. This means the only way they interact is via the weak interaction. This makes neutrinos are very difficult to detect. In particular, they pass through the CMS detector effectively without interacting at all, and their presence and the energy they carry can only be estimated using imbalance in transverse momentum of observed particles (see section~\ref{mt_met_recon}). 

Quarks come in two generations: up-type and down-type. The up-type quarks are the up quark (u), charm quark (c) and top quark (t). Their down-type counterparts are down quark (d), strange quark (s) and bottom quark (b). Each up-type quark carries a negative electric charge of 2/3 times the charge of the electron. Each down-type quark carries a negative charge of 1/3 times the charge od the electron. Just like the leptons, each progressive generation is heavier with the third generation consisting of the top and bottom quarks being the heaviest. In fact, the top quark was the last of the SM fermions to be discovered in 1995, and is the heaviest particle in the SM ($\sim 173\,\mathrm{GeV}/c^2$). As mentioned above, all quarks carry color charge. Color charge is to strong force as electric charge is to electromagnetic force. This allows quarks to interact via the strong force. Due to a phenomenon called color confinement, quarks aggregate together into colorless (having zero color charge) particles called hadrons. Hadrons are either formed of 3 (anti-)\,quarks (baryons) or 2 (anti-)\,quarks (mesons). The proton and neutron are baryons. It is made of two up quarks, and one down quark. It has a mass of $\sim 938.3\,\mathrm{MeV}/c^2$ and is stable (infinite lifetime). The neutron is made of one up quar and two down quarks. It has a mass of $\sim 939.5\,\mathrm{MeV}/c^2$ and has a lifetime of $\sim880\,$s.

Each particle described above has an anti-particle associated with it. Particles (matter) and their anti-paritcles (anti-matter) are almost identical except they have opposite physical charges (electric charge, color charge). For example, The anti-particle of an electron is the positron which is nearly identical to the electron except for the fact that it has positive electric charge.

The bosons in SM are carriers or mediators of force. Their behavior follow bose-einstein statistics and they are not constrained by the Pauli exclusion principle. The strong interaction, as its name suggests, is the strongest of the fundamental forces. The eight gluons mediate the strong interactions between particles with color charge. Photons are the mediators of the next strongest fundamental forece, the electromagnetic force. Gluons and photons are massless, electrically neutral and have spin 1. Additionally, gluons carry color charge. This is in contrast to photons which are electrically neutral. The $\mathrm{W}^+$,$\mathrm{W}^-$ and Z gauge bosons mediate the weak interactions between particles of different flavors. Both bosons have spin 1. However, unlike the photons and the gluons, they are heavy. The W boson has a mass of $\sim 80.4\,\mathrm{GeV}/c^2$ and the Z boson has a mass of $\sim 91.2\,\mathrm{GeV}/c^2$. Finally, the Higgs boson which is a massive, scalar (spin 0) and electrically neutral boson is responsible for giving masses to W, Z bosons and fermions.         

\subsection{Theory of interactions}

\section{Physics beyond the standard model}
\label{sec:BSM}

