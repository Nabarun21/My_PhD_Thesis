
%
% Chapter 1
%

\chapter{Introduction}

The Standard Model (SM) of particles physics is the most well-tested and elegant description of nature available
today. The discovery of the Higgs Boson in 2012~\cite{Aad:2012tfa, Chatrchyan:2012ufa, Chatrchyan:2013lba} added another feather in the hat of the SM. In the SM, elementary particles acquire mass from their interaction with the scalar Higgs field, the quantum of which is the Higgs Boson (h). This particle which had eluded particle physicists for years is a cornerstone of the SM, and in a way, was the last predicted missing piece associated with it. It was introduced, in 1964 by Brout, Englert, Higgs, Guralnik, Hagen and Kibble as a consequence of the electroweak symmetry breaking, in order to explain how elementary particles could obtain a mass without violating the gauge invariance of the SM~\cite{Englert:1964et,Higgs:1964ia,Higgs:1964pj,Guralnik:1964eu}.

It was nearly 50 years before the particle was discovered. During this period many important discoveries such as W/Z bosons (1983 at UA1/UA2 collaborations at CERN) and the top quark (1995 CDF/D0 at FermiLab) were made. The excellent performance of the Large Hadron Collider (LHC) at CERN (European Organization of Nuclear Research) in delivering proton-proton collisions, and the excellent work by the CMS (Compact Muon Solenoid) and ATLAS (A Large Toroidal Apparatus) collaborations made possible the discovery of the Higgs Boson in 2012. Although the CMS and the ATLAS are large general purpose detectors aimed at studying a wide range of physics, the discovery of the Higgs was one of their primary aims. They started collecting data in 2010, and the h discovery was made using the data collected from 2010 to 2012. In 2013, Peter Higgs and François Englert, two of the physicists associated with the development the theory, were jointly awarded the Nobel Prize in Physics.

This discovery was a signficant step for particle physics, and while it put an end to the decades old search for the elusive h, it opened up a fertile sector for particle physicists to explore and understand. One of the very important tasks is for particle physicists is to study if the properties of the discovered h are indeed compatible with theoretical SM expectations. In fact many such studies since 2012, have found properties of the h such as the spin, couplings, and charge-parity (CP) assignment to be consistent with SM~\cite{JHEP2016:45}. While more precise studies of the properties and couplings of the h is important, this discovery also provides us a portal to look for new physics Beyond the Standard Model (BSM). The SM, as mentioned above, is a remarkable theory that has stood the test of time. However, it is has its shortcomings and is not a complete theory. For example, the SM does not explain gravity and thus is inadequate as a candidate for an ideal ``Theory of Everything''. To address these shortcomings, many BSM theories have been proposed that modify the SM such that they are consistent with existing data but try to address its imperfections. Many outcomes these theories predict are non-SM and therecently discovered h unlocks a pristine ground to look for these outcomes. In fact, the constraint on the branching fraction to non-SM decay modes of the h, derived from a combined study by CMS and ATLAS is B(non-SM) $<$ 34\% at 95\% confidence level (CL)~\cite{JHEP2016:45}. Thus, a significant contribution from exotic (non-SM) decays is allowed in the BSM Higgs sector.

One such interesting process that is forbidden in the SM but occur in many new physics scenarios is interactionsbetween charged leptons that violate the conservation of Lepton Flavor. In particular, Lepton Flavor Violating (LFV) decays of the h are allowed by these theories, and could be realized in decays of the h which is neutral into two charged leptons of different flavor. In this dissertation, we describe a search looking for the decay of the h into a muon ($\mu$) and a tau lepton ($\tau$). Indirect constraints on $h \rightarrow \mu\tau$ exist through interpretations of measurements of processes such as $\tau \rightarrow \mu \gamma$~\cite{d}. These constraints set weak limits on such decays allowing significant branching fractions;$Br(H\rightarrow \mu \tau)<O(10\%).$~\cite{e}. A search for $H\rightarrow \mu \tau$ performed with data collected by the CMS during run I of the LHC  improved the above limits by an order of magnitude to $Br(H\rightarrow \mu \tau)<O(1.51\%)$ at 95\% confidence level ~\cite{df}. Also, an excess of events with a significance of 2.4$\sigma$ was observed. This warrants us to do this search with much larger amount of data which would either lead us to confirm this excess or squash it and set much stricter limits on this process. The run II of LHC (see section~\ref{sec:lhccms})  provides us with such an opportunity to perform the search outlined in the following.   





