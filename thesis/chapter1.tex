%
% Chapter 1
%

\chapter{Introduction}

The Standard Model (SM) is the most well-tested and elegant description of nature available today. The discovery of the Higgs Boson in 2012~\cite{Aad:2012tfa, Chatrchyan:2012ufa, Chatrchyan:2013lba} added another feather in the hat of the SM. In the SM, elementary particles acquire mass from their interaction with the scalar Higgs field, the quantum of which is the Higgs Boson (h). This particle which had eluded particle physicists for years is a cornerstone of the SM, and in a way, was the last predicted missing piece associated with it. It was introduced, in 1964 by Brout, Englert, Higgs, Guralnik, Hagen and Kibble as a consequence of the electroweak symmetry breaking, in order to explain how elementary particles could have mass without violating the gauge invariance of the SM~\cite{Englert:1964et,Higgs:1964ia,Higgs:1964pj,Guralnik:1964eu}.

It was nearly 50 years before the h was discovered. During this period many important discoveries such as W/Z bosons (1983 at UA1/UA2 collaborations at CERN) and the top quark (1995 CDF/D0 at FermiLab) were made. The excellent performance of the Large Hadron Collider (LHC) at CERN (European Organization of Nuclear Research) in delivering proton-proton collisions, and the excellent work by the CMS (Compact Muon Solenoid) and ATLAS (A Large Toroidal Apparatus) collaborations made possible the discovery of the Higgs Boson in 2012. Although the CMS and the ATLAS are large general purpose detectors aimed at studying a wide range of physics, the discovery of the Higgs was one of their primary aims. They started collecting data in 2010, and the h discovery was made using the data collected from 2010 to 2012. In 2013, Peter Higgs and François Englert, two of the physicists associated with the development of the theory, were jointly awarded the Nobel Prize in Physics.

This discovery was a signficant step for particle physics, and while it put an end to the decades old search for the elusive h, it opened up a fertile sector for particle physicists to explore and understand. One of the very important tasks is to ascertain if the properties of the discovered h are indeed compatible with theoretical SM expectations. In fact, many studies since 2012 have found properties of the h such as the spin, couplings, and charge-parity (CP) assignment to be consistent with SM~\cite{JHEP2016:45}. While more precise studies of the properties and couplings of the h is important, it also provides us with a portal to look for new physics Beyond the Standard Model (BSM). The SM, as mentioned above, is a remarkable theory that has stood the test of time. However, it is has its shortcomings and is not a complete theory. For example, the SM does not explain gravity and thus is inadequate as a candidate for an ideal ``Theory of Everything''. To address such shortcomings, many BSM theories have been proposed that modify the SM in such a way that they are consistent with existing observations, but at the same time try to address its imperfections. Many outcomes these theories predict are non-SM and the recently discovered h unlocks a pristine ground to look for these outcomes. In fact, the constraint on the branching fraction to non-SM decay modes of the h, derived from a combined study by CMS and ATLAS is B(non-SM) $<$ 34\% at 95\% confidence level (CL)~\cite{JHEP2016:45}. Thus, a significant contribution from exotic (non-SM) decays is allowed in the BSM Higgs sector.

One such interesting process that is forbidden in the SM but occurs in many new physics scenarios is interactions between charged leptons that violate the conservation of Lepton Flavor. In particular, Lepton Flavor Violating (LFV) decays of the h are allowed by these theories, and could be realized in decays of the h, which is neutral, into two charged leptons of different flavor. In this dissertation, we describe a search looking for LFV decay of the h into a muon ($\mu$) and a tau lepton ($\tau$). The tau lepton is short-lived and can further decay hadronically ($\tauh$) or into a electron. Since we can detect electrons better than tau leptons, the latter channel has a cleaner signature. In particular, the search described here looks for this electronic channel signature of a LFV decay of h boson, i.e. $\hmue$. Indirect constraints on $h \rightarrow \mu\tau$ exist through interpretations of measurements of processes such as $\tau \rightarrow \mu \gamma$~\cite{kanemura}. These constraints set weak limits on such decays allowing significant branching fractions; $Br(H\rightarrow \mu \tau)<O(10\%)$~\cite{Blankenburg:2012ex,Harnik:2012pb}. A search was performed by CMS for $H\rightarrow \mu \tau$ with proton-proton collision data at center-of-mass energy of 8\,TeV, collected during run I (2010-12) of the LHC. This improved the above limits by an order of magnitude to $Br(H\rightarrow \mu \tau)<O(1.51\%)$ at 95\% confidence level~\cite{Khachatryan:2015kon}. However, an excess of events with a significance of 2.4\,$\sigma$ was also observed. This warrants us to do this search with a larger amount of data which would either lead us to confirm this excess, or squash it and set much stricter limits on this process. The dataset collected by the CMS detector in 2016 provides us with such an opportunity. It corresponds to proton-proton collision data at a much higher center-of-mass energy of 13\,TeV and is almost two times in size of the run I dataset. Besides using this larger dataset, the analysis described in this thesis improves upon previous searches by introducing multivariate techniques.

An interesting common feature of many of the models that allow LFV decays of the h is that they predict the existence of heavy neutral higgs bosons, H(CP-even) and A(CP-odd). These are also expected to have LFV decays into charged leptons of different flavor~\cite{PhysRevD.93.055021}. A direct search for these channels would thus provide a complementary probe of these models. In this dissertation, we also describe such a search for heavy neutral higgs boson (H) deacying in a lepton flavor violating manner into a muon and an electronically decaying tau, .i.e $\Hmue$. For this search, we probe H mass ($m_H$) in the range $200<m_H<900$\,GeV, and use analysis techniques similar to the $\hmue$ search. This search is the first ever direct search to look for this process. In this entire document, we denote neutral heavy Higgs boson simply by H and SM Higgs boson as h.

The dissertation is devoted to the description of the $\hmue$ and $\Hmue$ searches using the CMS experiment at the LHC. In chapter~\ref{chap:theory}, we describe theoretical background and motivations for these searches. In the next chapter (~\ref{chap:exper_setup}), we describe the experimental apparatus used for the search, i.e. the collider (LHC) and the detector (CMS). In the following chapter (~\ref{chap:event_sim}), the procedure for simulation of events and reconstruction of physics objects such as electrons, muons and jets are outlined. Chapter~\ref{evt_sel} describes the strategies followed to select events with the signal signature, and to increase the percentage of signal-like events in the sample thereby increasing the sensitivity of the searches. In chapter~\ref{bg_val}, estimation of background processes for both searches is outlined. Chapter~\ref{sig_ext} provides a description of the statistical methods used for signal extraction and setting of exclusion limits, and also the uncertainties associated with the searches. Finally, chapter~\ref{results} lays out the results of both the searches performed.    





