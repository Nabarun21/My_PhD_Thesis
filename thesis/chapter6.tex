
%
% Chapter 6
%

\chapter{Search for LFV decays of h125}
\label{chap:analysis_smhiggs}
This chapter describes the analysis in search for lepton flavor violating decay of the h125 boson into a muon and an electronically decaying tau lepton. The motivation for this search is discussed in the introduction and theory sections. In the following sections the dataset and MC samples used, event selection and subsequent analysis strategy, systematic uncertainties affecting the analysis and finally the results are described in detail.

\section{Datasets and MC samples}

The data analysed in this search was gathered by the CMS detector in 2016 during proton-proton collisions at the LHC, corresponding to an integrated luminosity of 35.9 fb^{-1}. This data corresponds to a center-of-mass energy of 13 TeV and the spacing between bunch crossings in the LHC was 25ns with an average of about 30 collisions (see detailed discussion on pileup below) per bunch crossing. The subset of samples used among all collected by CMS are the ones having at least one isolated muon having transverse energy over 24 GeV, as triggered by the CMS high level isolated muon trigger (HLT_IsoMu24 in CMS parlance).

Monte Carlo simulated SM Higgs Boson events produced by  gluon fusion (GF), vector boson fusion (VBF) and associated production (production in association with a W or Z boson) mechanisms and decaying into a muon and a tau lepton are used as signal samples for the analysis.POWHEG1.0 is used to produce these samples. It is also used to simulate t\bar{t} and single-top quark production processes. Several other event generators were used to simulate various other background processes for the analysis.These include MADGRAPH5.1 for Z+Jets, W+Jets and W$\gamma$ processes and AMC@NLO for diboson production. All the event generators are interfaced to PYTHIA 8.1 for the showering of partons and hadronization, as well as including a simulation of the underlying event (UE) and multiple interaction (MPI) based on the CUET8PM1 tune. After the generation step, the detector response is simulated using a detailed description of the simulated detector based on the GEANT4 package.












% % uncomment the following lines,
% if using chapter-wise bibliography
%
% \bibliographystyle{ndnatbib}
% \bibliography{example}
