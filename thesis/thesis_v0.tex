
%%%%%%%%%%%%%%%%%%%%%%%%%%%%%%%%%%%%%%%%%%%%%%%%%%%%%%%%%%%%%%%%%%%%%%%%
%
% You should *also* have a ND formatting guide to ensure that you have
% all the relevant parts, put the captions in the right place, etc.
% Just because you have this wonderful style classfile doesn't mean
% that it removes *all* the formatting onus from you.  :-)
% Although be warned that the Graduate School has been known to let
% their official formatting guide get out of date. When in doubt,
% the Microsoft Word example seemed to be the only thing kept
% consistently up-to-date in 2013, and is probably the safest thing
% to consult.
%
% You should break all of this stuff up into separate files
% (at the very least, one chapter per file) and use the \include
% command, as has been done here for chapters 1 and 2 and the appendix.
% There is also an \input command, but \include is more commonly used to
% import chapters in books and dissertations. For the differences between these
% two commands, see, e.g., 
% http://web.science.mq.edu.au/~rdale/resources/writingnotes/latexstruct.html
% or http://tex.stackexchange.com/questions/246/when-should-i-use-input-vs-include.
%
% If you compile from the command line, note that you should also have 
% a good Makefile; one that invokes LaTeX as many times as necessary 
% (up to 4) and bibtex if necessary.
%
% If you use an editor that allows you to compile from within the
% program, note that you will need to compile up to four times. Also,
% we recommend that you use pdflatex (sometimes displayed as
% LaTeX => PDF) to compile directly to pdf.
%
% If you have any suggestions, comments, questions, please send e-mail
% to: dteditor@nd.edu
%
%%%%%%%%%%%%%%%%%%%%%%%%%%%%%%%%%%%%%%%%%%%%%%%%%%%%%%%%%%%%%%%%%%%%%%%%

\documentclass[final,numrefs,sort&compress]{nddiss2e}
\usepackage{ mathrsfs }
\usepackage{enumitem}
\usepackage{subfigure}
\usepackage{multirow}
\usepackage{ amssymb }
% \setlist[itemize]{leftmargin=45mm}
% One of the options draft, review, final must be chosen.
% One of the options textrefs or numrefs should be chosen
% to specify if you want numerical or ``author-date''
% style citations.
% Other available options are:
% 10pt/11pt/12pt (available with draft only)
% twoadvisors
% noinfo (should be used when you compile the final time
%         for formal submission)
% sort (sorts multiple citations in the order that they're
%       listed in the bibliography)
% compress (compresses numerical citations, e.g. [1,2,3]
%           becomes [1-3]; has no effect when used with
%           the textrefs option)
% sort&compress (sorts and compresses numerical citations;
%           is identical to sort when used with textrefs)

\begin{document}

\newcommand{\PH}{\ensuremath{\text{H}}}
\newcommand{\Ph}{\ensuremath{\text{h}}}
\newcommand{\Pgm}{\ensuremath{\mu}}
\newcommand{\Pe}{\ensuremath{\text{e}}}
\newcommand{\Pgt}{\ensuremath{\tau}}
\newcommand{\pt}{\ensuremath{\text{p}_\text{T}}}
\newcommand{\PW}{\ensuremath{\text{W}}}
\newcommand{\PZ}{\ensuremath{\text{Z}}}
\newcommand{\GeV}{\ensuremath{\,\text{Ge\hspace{-.08em}V}}\xspace}
\newcommand{\ptvec}{\ensuremath{{\vec p}_{\mathrm{T}}}\xspace}
\newcommand{\ptmiss}{\ensuremath{\pt^\text{miss}}\xspace}
\newcommand{\ptvecmiss}{\ensuremath{{\vec p}_{\mathrm{T}}^{\kern1pt\text{miss}}}\xspace}
\newcommand{\dphiemet}{\ensuremath{\Delta\phi(\Pe, \ptvecmiss)}}
\newcommand{\dphimumet}{\ensuremath{\Delta\phi(\Pgm, \ptvecmiss)}}
\newcommand{\dphiemu}{\ensuremath{\Delta\phi(\Pe, \Pgm)}}
\newcommand{\Htetmu}{\ensuremath{\PH \to \Pgt_{\Pe} \Pgt_{\Pgm}}\xspace}
\newcommand{\Hteth }{\ensuremath{\PH \to \Pgt_{\Pe} \Pgt_{\textrm{h}}}\xspace}
\newcommand{\Hmt}{\ensuremath{\PH \to \Pgm \Pgt}\xspace}
\newcommand{\met}{\ensuremath{\cancel{\it{E}}_{T}}\xspace}
\newcommand{\msig}{\ensuremath{100\:\GeV< M_{collinear} < \: 150\:\GeV}\xspace}
\newcommand{\tauh}{\ensuremath{\Pgt_{h}}\xspace}
\newcommand{\Htt}{\ensuremath{\Ph \to \Pgt \Pgt}\xspace}
\newcommand{\ztt}{\ensuremath{\PZ \to \Pgt \Pgt}\xspace}
\newcommand{\Het}{\ensuremath{\Ph \to \Pe \Pgt}\xspace}
\newcommand{\hmue}{\ensuremath{\Ph \to \Pgm \Pgt_{e}}\xspace}
\newcommand{\Hmue}{\ensuremath{\PH \to \Pgm \Pgt_{e}}\xspace}
\newcommand{\hmu}{\ensuremath{\Ph \to \Pgm \Pgt}\xspace}
\newcommand{\Hmu}{\ensuremath{\PH \to \Pgm \Pgt}\xspace}
\newcommand{\Hmuhad}{\ensuremath{\PH \to \Pgm \Pgt_{\text{h}}}\xspace}
\newcommand{\hmuhad}{\ensuremath{\Ph \to \Pgm \Pgt_{\text{h}}}\xspace}
\newcommand{\Hehad}{\ensuremath{\PH \to \Pe \Pgt_{\text{h}}}\xspace}
\newcommand{\Hemu}{\ensuremath{\PH \to \Pe \Pgt_{\Pgm}}\xspace}
\newcommand{\mue}{\ensuremath{\Pgm \Pgt_{\Pe}}\xspace}
\newcommand{\emu}{\ensuremath{\Pe \Pgt_{\Pgm}}\xspace}
\newcommand{\muhad}{\ensuremath{\Pgm \Pgt_{\text{h}}}\xspace}
\newcommand{\ehad}{\ensuremath{\Pe \Pgt_{\text{h}}}\xspace}
\newcommand{\mcol}{\ensuremath{M_{\text{col}}}\xspace}
\newcommand{\mvis}{\ensuremath{M_{\text{vis}}}\xspace}
\newcommand{\wjets}{\ensuremath{\PW+\text{jets}}\xspace}
\newcommand{\zjets}{\ensuremath{\PZ+\text{jets}}\xspace}
\newcommand{\aMCATNLO} {\textsc{MG5}\_a\MCATNLO\xspace}
\newcommand{\ttb}{\ensuremath{t\overline{t}}\xspace}

\frontmatter % All the items before the first chapter go in ``frontmatter''

% Titles may be 1-4 lines long. If your title is longer than 4 lines,
% the class file may have difficulty formatting the title page.
% Line-breaks in the title have to be protected with `\protect`.
\title{Search for lepton flavor violating decays \protect\\ of Higgs Bosons \protect\\ with the CMS experiment}
\author{Nabarun Dev}
\work{Dissertation} % or \work{Thesis}
\degaward{Doctor of Philosophy \\ in \\Physics} % or 
%\degaward{Master of Science \\ in \\ Subject}
\advisor{Colin Philip Jessop}
\department{Physics}

\maketitle
%%%%%%%%%%%%%%%%%%%%%%%%%%%%%%%%%%%%%%%%%%%%%%%%%%%%%%%%%%%%%%%%%%%%%%%%
%
% Front stuff
%
%%%%%%%%%%%%%%%%%%%%%%%%%%%%%%%%%%%%%%%%%%%%%%%%%%%%%%%%%%%%%%%%%%%%%%%%

\makepublicdomain

% An abstract is optional for a mster's thesis, and required for a doctoral dissertation.
\begin{abstract}
\end{abstract}

% A dedication is optional.
\renewcommand{\dedicationname}{Dedicated to}

\begin{dedication}
  To my family
\end{dedication}

% These are required, and must be in this order.
\tableofcontents
\listoffigures
\listoftables

% A preface is optional.
\begin{preface}
  Long time ago in a galaxy far far away....(preface is optional)
\end{preface}

% It's hard to tell from the information available from the Graduate
% School in Spring 2013 whether or not an acknowledgements section is optional.
\begin{acknowledge}
  I would like to acknowledge the light side of the force, Master Kenobi and Grand Master Yoda.
\end{acknowledge}

% A symbols section is optional.
\begin{symbols}
  \sym{c}{speed of light}
  \sym{m}{mass}
  \sym{e}{elementary charge}
  \sym{E}{energy}  
\end{symbols}


\mainmatter
% Place the text body here.
%\include{chapter-one}
%Begin each chapter with \chapter{Title}. Both the thesis title and
%chapter titles should match in style.

%
% An unnumbered chapter (features)
%

%
% Chapter 1
%

%%
% Chapter 1
%

\chapter{Introduction}
\epigraph{Modern science has been a voyage into the unknown, with a lesson in humility waiting at every stop. Many passengers would rather have stayed home.}{\textit{Carl Edward Sagan}}
\vskip 0.6in
The Standard Model (SM) is the most well-tested and elegant description of nature available today. The discovery of the Higgs Boson in 2012~\cite{Aad:2012tfa, Chatrchyan:2012ufa, Chatrchyan:2013lba} completed the SM. In the SM, elementary particles acquire mass from their interaction with the scalar Higgs field, the quantum of which is the Higgs Boson (h). This particle which had eluded physicists for years is a cornerstone of the SM, and in a way, was the last predicted missing piece associated with it. It was introduced, in 1964 by Brout, Englert, Higgs, Guralnik, Hagen and Kibble as a consequence of electroweak symmetry breaking, in order to explain how elementary particles could have mass without violating the gauge invariance of the SM~\cite{Englert:1964et,Higgs:1964ia,Higgs:1964pj,Guralnik:1964eu}.

It was nearly 50 years before the h was discovered. During this period many important discoveries such as W/Z bosons (1983 at UA1/UA2 collaborations at CERN) and the top quark (1995 CDF/D0 at FermiLab) were made. The excellent performance of the Large Hadron Collider (LHC) at CERN (European Organization of Nuclear Research) in delivering proton-proton collisions, and the excellent work by the CMS (Compact Muon Solenoid) and ATLAS (A Large Toroidal Apparatus) collaborations made possible the discovery of the Higgs Boson in 2012. Although the CMS and the ATLAS are large general purpose detectors aimed at studying a wide range of physics, their design was optimized to study the phenomenon of Electroweak Symmetry Breaking, making the discovery of the Higgs one of their primary aims. They started collecting data in 2010, and the h discovery was made using the data collected from 2010 to 2012. In 2013, Peter Higgs and François Englert, two of the physicists associated with the development of the theory, were jointly awarded the Nobel Prize in Physics.

This discovery was a significant step for particle physics, and while it put an end to the decades old search for the elusive h, it opened up a fertile sector for particle physicists to explore and understand. One of the very important tasks is to ascertain if the properties of the discovered h are indeed compatible with theoretical SM expectations. In fact, many studies since 2012 have found properties of the h such as the spin, couplings, and charge-parity (CP) assignment to be consistent with SM~\cite{JHEP2016:45}. While more precise studies of the properties and couplings of the h is important, it also provides us with a portal to look for new physics Beyond the Standard Model (BSM). The SM, as mentioned above, is a remarkable theory that has stood the test of time. However, it is has its shortcomings and is not a complete theory. For example, the SM does not explain gravity and thus is inadequate as a candidate for an ideal ``Theory of Everything''. It also doesn't explain why the weak force is $10^{24}$ times stronger than gravity. This is usually known as the hierarchy problem. To address such shortcomings, many BSM theories have been proposed that modify the SM in such a way that they are consistent with existing observations, but at the same time try to address its imperfections. Many outcomes these theories predict are non-SM and the recently discovered h unlocks a pristine ground to look for these outcomes. In fact, the constraint on the branching fraction to non-SM decay modes of the h, derived from a combined study by CMS and ATLAS is B(non-SM) $<$ 34\% at 95\% confidence level (CL)~\cite{JHEP2016:45}. Thus, a significant contribution from exotic (non-SM) decays is allowed in the BSM Higgs sector.

One such interesting process that is forbidden in the SM but occurs in many new physics scenarios is interactions between charged leptons that violate the conservation of Lepton Flavor. In particular, Lepton Flavor Violating (LFV) decays of the h are allowed by these theories, and could be realized in decays of the h, which is neutral, into two charged leptons of different flavor. In this dissertation, we describe a search looking for LFV decay of the h into a muon ($\mu$) and a tau lepton ($\tau$). The tau lepton is short-lived and can further decay hadronically ($\tauh$) or into a electron. Since we can detect electrons better than tau leptons, the latter channel has a cleaner signature. In particular, the search described here looks for this electronic channel signature of a LFV decay of h boson, i.e. $\hmue$. Indirect constraints on $\hmu$ exist through interpretations of measurements of processes such as $\tau \rightarrow \mu \gamma$~\cite{kanemura}. These constraints set weak limits on such decays allowing significant branching fractions; $Br(\hmu)<O(10\%)$~\cite{Blankenburg:2012ex,Harnik:2012pb}. A search was performed by CMS for $\hmu$ with proton-proton collision data at center-of-mass energy of 8\,TeV, collected during run I (2010-12) of the LHC. This improved the above limits by an order of magnitude to $Br(\hmu)<O(1.51\%)$ at 95\% confidence level~\cite{Khachatryan:2015kon}. However, an excess of events with a significance of 2.4\,$\sigma$ was also observed. This warrants us to do this search with a larger amount of data which would either lead us to confirm this excess, or squash it and set much stricter limits on this process. The dataset collected by the CMS detector in 2016 provides us with such an opportunity. It corresponds to proton-proton collision data at a much higher center-of-mass energy of 13\,TeV and is almost two times in size of the run I dataset. Besides using this larger dataset, the analysis described in this thesis improves upon previous searches by introducing multivariate techniques.

An interesting common feature of many of the models that allow LFV decays of the h is that they predict the existence of heavy neutral Higgs bosons, H (CP-even) and A (CP-odd). These are also expected to have LFV decays into charged leptons of different flavor~\cite{PhysRevD.93.055021}. A direct search for these channels would thus provide a complementary probe of these models. In this dissertation, we also describe such a search for heavy neutral Higgs boson (H) decaying in a lepton flavor violating manner into a muon and an electronically decaying tau, .i.e $\Hmue$. For this search, we probe H mass ($m_H$) in the range $200<m_H<900$\,GeV, and use analysis techniques similar to the $\hmue$ search. This search is the first ever direct search to look for this process. In this entire document, we denote neutral heavy Higgs boson simply by H and SM Higgs boson as h.

The dissertation is devoted to the description of the $\hmue$ and $\Hmue$ searches using the CMS experiment at the LHC. In chapter~\ref{chap:theory}, we describe theoretical background and motivations for these searches. In the next chapter (\!~\ref{chap:exper_setup}), we describe the experimental apparatus used for the search, i.e. the collider (LHC) and the detector (CMS). In the following chapter (\!~\ref{chap:event_sim}), the procedure for simulation of events and reconstruction of physics objects such as electrons, muons and jets are outlined. Chapter~\ref{evt_sel} describes the strategies followed to select events with the signal signature, and to increase the percentage of signal-like events in the sample thereby increasing the sensitivity of the searches. In chapter~\ref{bg_val}, estimation of background processes for both searches is outlined. Chapter~\ref{sig_ext} provides a description of the statistical methods used for signal extraction and setting of exclusion limits, and also the uncertainties associated with the searches. Finally, chapter~\ref{results} lays out the results of both the searches performed.    








%
% Chapter 2
%

%
%
% Chapter 2
%

\chapter{Theoretical bases}
\label{chap:theory}
In this chapter we describe the theoretical motivations that drive the searches described in this thesis. We start with a description  the standard model (SM), its particle content and interactions and the Higgs mechanism. We then talk about the inadequacies of the SM, and the existence of physics beyond the standard model (BSM). We then outline a few BSM models and how they point towards the possibility of the decays that we search for in this thesis.  

\section{The Standard Model }
\label{sec:SM}
The SM is the result of human endeavors over centuries to understand what we and the world around us are made of, and capture those ideas in beautiful mathematical form. Our understanding of the world around us has refined progressively from the ancient times, when best tools of observation we had were nothing but our own eyes to the current day when we are able we collide particles that make up matter at unprecedented speeds, and have sophisticated tools like the CMS detector to aid us. From the ancient greeks who pondered over philosophical questions about what the basic elements of nature were, to the discovery of electron in 1898 by J.J.Thompson, to Rutherford's famous gold foil experiment, to the discovery of the neutron by James Chadwick in 1932 have been stepping stones towards our understanding of nature and the formulation of SM. During the course of its formulation and after, the SM has accurately explained phenomena alreafy known and  predicted the existence of particles that were discovered later. The last of these particles is the Higgs Boson, discovered in 2012 at CERN by the CMS and the ATLAS experiments~\cite{Aad:2012tfa, Chatrchyan:2012ufa, Chatrchyan:2013lba}. The SM is a gauge theory, in which three of the four known natural forces (strong, electromagnetic, weak and not gravity~\ref{tab:forces}) are represented by the SU(3)xSU(2)xU(1) symmetry group. This symmetry group describes under which transformations the SM is invariant. By Noether's theorem each of the above symmetries associated with the SM Lagrangian is associated with a conserved quantity: color charge, weak isospin and electric charge. The following describes the elementary particles of the SM, the interactions among these and finally, the spontaneous symmetry breaking mechanism.

\subsection{Elementary particles}
There are two kinds of elementary particles in the SM. They are characterized by the intrinsic angular momentum that they carry, i.e. by their spin. Fermions, which has half-integer spins, form the building blocks of matter. Bosons, which have integer spins, are the force-carriers or mediators of interactions.

Fermions are fundamental particles, i.e. they cannot be broken down into further consituents. The space-time evolution of the fermions is described by the Dirac equation and their behavior follows Fermi-Dirac statistics. All fermions are subject to the Pauli exclusion principle. The fermions can be further categorized into two classes depending on their interaction with the strong force. Fermions wich do not interact with the strong force are called leptons, and do not carry any color charge. Quarks carry color charge and interact via the strong force. Both leptons and quarks are further classified into three generations. Each lepton generation consists of a lepton and a neutrino while each quark generation consists of a up type and a down type quark. These are outlined in detail below.

Leptons comprise of the familiar electron (e), its heavier cousins muon ($\Pgm$) and tau lepton ($\Pgt$) which carry the same negative electric charge as the electron ($1.6\times10^{-19} C$).  The heavier leptons $\Pgt$ ($\sim 1.8\,\mathrm{GeV}/c^2$ ) and $\Pgm$ ($\sim 105.7\,\mathrm{MeV}/c^2$) have short lifetimes of $\sim 2.9\times 10^{-13}\,$s and $\sim 2.2\times 10^{-6}\,$s respectively. The eventually decay into an electron which is the lightest lepton ($\sim 0.5\,\mathrm{MeV}/c^2$ ) and has infinite lifetime, or lighter hadrons. In the CMS detector, the $\mu$ survives long enough to reach the muon systems is thus detected as its own distinct signature. The $\Pgt$ on the other hand, owing to its extremely short lifetime, can travel only a very short distance ($\sim <10\,mm$) before decaying. Thus, only decay products of tau leptons are able to be directly detected by CMS. Each charged lepton is associated with an electrically neutral neutrino. They are called electron neutrino ($\nu_e$), muon neutrino ($\nu_{\mu}$) and tau neutrino ($\nu_{\mu}$). Because neutrinos carry no electric charge, they do not interact via electromagnetic interaction. This means the only way they interact is via the weak interaction. This makes neutrinos are very difficult to detect. In particular, they pass through the CMS detector effectively without interacting at all, and their presence and the energy they carry can only be estimated using imbalance in transverse momentum of observed particles (see section~\ref{mt_met_recon}). 

Quarks come in two generations: up-type and down-type. The up-type quarks are the up quark (u), charm quark (c) and top quark (t). Their down-type counterparts are down quark (d), strange quark (s) and bottom quark (b). Each up-type quark carries a negative electric charge of 2/3 times the charge of the electron. Each down-type quark carries a negative charge of 1/3 times the charge od the electron. Just like the leptons, each progressive generation is heavier with the third generation consisting of the top and bottom quarks being the heaviest. In fact, the top quark was the last of the SM fermions to be discovered in 1995, and is the heaviest particle in the SM ($\sim 173\,\mathrm{GeV}/c^2$). As mentioned above, all quarks carry color charge. Color charge is to strong force as electric charge is to electromagnetic force. This allows quarks to interact via the strong force. Due to a phenomenon called color confinement, quarks aggregate together into colorless (having zero color charge) particles called hadrons. Hadrons are either formed of 3 (anti-)\,quarks (baryons) or 2 (anti-)\,quarks (mesons). The proton and neutron are baryons. It is made of two up quarks, and one down quark. It has a mass of $\sim 938.3\,\mathrm{MeV}/c^2$ and is stable (infinite lifetime). The neutron is made of one up quar and two down quarks. It has a mass of $\sim 939.5\,\mathrm{MeV}/c^2$ and has a lifetime of $\sim880\,$s.

Each particle described above has an anti-particle associated with it. Particles (matter) and their anti-paritcles (anti-matter) are almost identical except they have opposite physical charges (electric charge, color charge). For example, The anti-particle of an electron is the positron which is nearly identical to the electron except for the fact that it has positive electric charge.

The bosons in SM are carriers or mediators of force. Their behavior follow bose-einstein statistics and they are not constrained by the Pauli exclusion principle. The strong interaction, as its name suggests, is the strongest of the fundamental forces. The eight gluons mediate the strong interactions between particles with color charge. Photons are the mediators of the next strongest fundamental forece, the electromagnetic force. Gluons and photons are massless, electrically neutral and have spin 1. Additionally, gluons carry color charge. This is in contrast to photons which are electrically neutral. The $\mathrm{W}^+$,$\mathrm{W}^-$ and Z gauge bosons mediate the weak interactions between particles of different flavors. Both bosons have spin 1. However, unlike the photons and the gluons, they are heavy. The W boson has a mass of $\sim 80.4\,\mathrm{GeV}/c^2$ and the Z boson has a mass of $\sim 91.2\,\mathrm{GeV}/c^2$. Finally, the Higgs boson which is a massive, scalar (spin 0) and electrically neutral boson is responsible for giving masses to W, Z bosons and fermions.         

\subsection{Theory of interactions}

\section{Physics beyond the standard model}
\label{sec:BSM}



%
% Chapter 3
%

%\chapter{Experimental Setup}
\label{chap:exper_setup}
..introduce...



%
%
% Chapter 4
%

\chapter{Object reconstruction and event generation}
\label{chap:event_sim}
\section{Introduction}
\label{intro}
This chapter is divided into two parts. In the first part, the procedure for the generation of simulated events is described. This is done in several distinct stages with the output of one stage serving as an input for the next. A suite of software packages, developed mostly by the particle and nuclear physics communities, is used to achieve this. This part concludes by detailing the simulated datasets used in the analyses described in this thesis. In the second part of this chapter, the reconstruction of physics objects is described in detail. It starts with a description of the particle-flow algorithm which is kind of a global event reconstruction scheme for the entire event. This is followed by descriptions of track , muon and electron reconstructions. Reconstruction of jets is described next followed by description of composite objects used in the analysis such as collinear mass and transverse mass. Brief desciptions of tau lepton reconstruction and b-tagging of jets is also included.


\section{Event Simulation}
A $pp$ collision at the LHC, like any hadronic collision, is more complex than the hard interaction of two participating partons. The proton being a composite object, the colliding partons from the hard interaction are accompanied by other quarks and gluons that interact and rearrange themseleves into colorless objects. A $pp$ collision thus consists of: the Hard Scattering which represents the part of the collision where two partons in the initial state interact by exchanging a high transverse momentum, and the Underlying Event that represent the interaction of the everything else in the collision except the partons in hard scattering. In addition to the implementing the above, i.e. physics of a $pp$ collision  that produces a bunch of final state particles, the event simulation also has to include interactions of these particles with the CMS detector. Monte Carlo methods, that use generation of random numbers to simulate sampling from a given probability distribution , are used to model the above event simulations~\cite{mc_evtsim}.

\subsection{Monte Carlo method}

Monte Carlo (MC) methods (named after a famous casino in the city state of Monaco) are a broad class of computational algorithms that rely on repeated random sampling to obtain numerical results~\cite{mcwiki}. In particle physics, these methods play a key role in generation of events and are used primary for : generation of samples from specified probabilty distributions, and the calculation of integrals. Programs which implement the above method, called MC event generators, use generation of random numbers to make decisions about physics processes. These can range from selection of processes are generated in the collision, to which decay channel a particle decays in, to making decisions on how the particle interacts with detector material. Usually, each such decision is the result of a draw from a distribution which depends only on the current state the process is in, and not on previous states. The MC generator is provided as input the distributions that represent the physics of the generated particles, their production, their decay modes and their couplings. A MC generator starts by using a pseudo-random number generator that usually outputs a random number between 0 and 1 with. Although, true random number generation can only be done by physical processess, modern pseudo-random number generators are known to generate numbers with a high degree of randommness. Starting from this distribution, the MC event generator uses one of the various methods such as the inverse-transform method, or the rejection sampling method to convert this uniform distribution into a desired probablity distribution, $p(x)$. It is then possible to generate random numbers according to this distribution to simulate physical processes. 


\subsection{CMS simulation pipeline}

The MC simulation of events in CMS consists of the following sequential steps. The first step is simulation of the Hard Scattering.As mentioned earlier, this represents the primary hard interaction in a collision where two partons in the initial state interact by exchanging high transverse momentum resulting in a final state with two or more partons. The parton density function (pdf) which parametrizes the distributions of the partons inside each hadron are used to model the momenta of incoming partons. It represents the probability of finding a parton of a certain flavour at a certain longitudinal momentum fraction, when the hadron, that contains it, is probed at a certain scale. The PDF are extracted from fits to the data, mainly from ep collisions, and various PDF sets are available for each parton flavour. Commonly used pdf sets include ones provided by the  CTEQ, HERA (H1 and ZEUS) and NNPDF collaborations. The LHAPDF library provides a unified C++ interface to all major PDF sets. The matrix element formulation is used to model the hard scattering process to leading order in perturbative QCD, or to higher orders depending on the generator. The next step is simulation of the parton shower. The hadronization and radiation of quarks and gluons in the initial and final states cannot be feasibly encapsulated in the matrix element computation. Parton shower describes these missing parts. The matrix element calculations are combined with the parton shower by one of the different matching schemes which ensure that there is no double counting of terms present in both the matrix element and the partion shower expansion. The matching schemes that are most often used are MLM~\cite{mlm}, CKKW~\cite{ckkw} and FxFx~\cite{Frederix:2012ps}. The simulation of the Underlying Event comes next. Underlying event includes everything in the collision that is not associated with the primary hard scattering process. They consist mostly of soft QCD interactions, and implemented using the MC event generators and interfaced with the matrix element simulation. The hadronization of the quarks and gluons is simulated next and it conisists of recombination of individual partons into colorless hadrons. Lastly the decay of short-lived particles is simulated.

An important part of the event generation chain is the simulation of pileup. The protons circulate inside the LHC not as a continuous beam but in discrete closely packed bunches. This leads to more than one proton-proton collision per bunch crossing, i.e. pileup both in-time and out-of-time (see chapter~\ref{chap:exper_setup}). Event generators add  pile-up events to the hard scattering samples by randomly simulating soft inelastic collisions and overlapping them. The distribution of the number of pileup interactions in data is hard to predict. MC event generators usually produce events for a scenario with a higher number of pileup vertices, and with a flat disctribution of number of vertices . This is afterwards reweighted to match the observed distribution of pileup interactions in data.

Several MC generators have been developed. Some of these can produce all components of the above simulation pipeline while some calculate only the matrix element and need to be interfaced with other generators for the simulation of remaining parts. Pythia~\cite{Sjostrand:pythia8} and Herwig~\cite{herwig} can produce the entire chain while Powheg~\cite{Nason:2004rx,Frixione:2007vw, Alioli:2010xd, Alioli:2010xa, Alioli:2008tz, Bagnaschi:2011tu}, aMC@NLO~\cite{Alwall:2014} and Madgraph~\cite{Alwall:2011uj} produce up to matrix element stage. Powheg and aMC@NLO can perform next-to-leading order calculations. 

Finally, the Geant4 (GEometry ANd Tracking)~\cite{GEANT4} package is used to simulate the interaction of physical particles after the collision, produced by pipeline described  above, with a sophisticated and complex simulation of the detector itself. This simulated detector response is used as input for the same physics reconstruction algorithms (desribed in the next section), that are used to reconstruct the data, thus enabling a direct comparison of the two. If differences are observed in the behavior of these reconstruction algorithms for MC events in comparsion to observed data, the MC events are tuned to the behavior observed in data. 


\section{MC samples used for the analyses}
\label{samples_mc}

The {ggH} and VBF Higgs boson samples are generated with POWHEG 2.0 while an extension of POWHEG 2.0~\cite{Luisoni:2013kna} is used for the $\PW\PH$ and $\PZ\PH$ simulated samples. For the \Hmue analysis, only the gluon fusion (ggH) production mode has been considered. Samples are generated for a range of H masses from 200 to 900 GeV.

The $\zjets$ and $\wjets$ processes are simulated using the \aMCATNLO generator at leading order (LO) with the MLM jet matching and merging scheme. The same generator is also used for diboson production which is simulated at  next-to-LO (NLO) with the FxFx jet matching and merging scheme. POWHEG 2.0 and 1.0 are used for top quark-antiquark ($\ttb$) and single top quark production, respectively. The POWHEG and MADGRAPH generators are interfaced with PYTHIA 8 for parton showering, fragmentation, and decays. 

As mentioned earlier in this chapter, additional pileup interactions are also a part of the MC generation pipeline. All simulated samples are reweighted to the pileup distribution observed in data. An event weight is applied based on the number of simulated pileup events and the instantaneous luminosity per bunch-crossing, averaged over the run period. Several other scale factors  are used to reweight the events in order to get the MC simulation to match the data closely. These include scale factors based on trigger, lepton identification, lepton isolaton and b-jet tagging efficiencies.

\section{Physics Object Reconstruction}
\label{p_ob_recon}
This section begins with the description of the particle-flow algorithm followed by reconstruction of tracks and vertices, electrons, muons, jets and other physics objects.  
\subsection{Particle Flow}
\label{p_flow}
\subsection{Track and primary vertex reconstruction}
\label{track_recon}

Tracks of charged particles, that traverse the CMS tracker (described in section~\ref{tracker}), are reconstructed~\cite{track_reconstruction} using hits from the pixel and strip detectors in the tracker. Hits are reconstructed by clustering signals above specified thesholds in the pixel and strip channels, and then estimating the cluster positions and uncertainties in a local orthogonal system plane of each sensor. During track reconstruction, a translation is made between the local coordinate system of these hits to the global coordinate system of the tracks. The software used to reconstruct tracks by CMS is called the Combinatorial Track Finder (CTF) and is adaptation of the Kalman filter~\cite{kalman_filter}. Tracks are reconstructed using a iterative procedure with the basic idea being, that tracks that are easiest to find (e.g., high \pt tracks, and produced near the interaction region) are searched in the initial iterations with subsequesnt iterations looking for more difficult sets of tracks (e.g., low \pt tracks , or tracks produced far from the interaction region). Hits unambiguously assigned to the track in the previous iterations are removed for the subsequent ones, thus reducing the combinatorial complexity. Each iteration can be divided into four sequential steps.

The first step is seed generation which provides initial track candidates that define the starting trajectory parameters and associated uncertainties of potential tracks. Charged particles follow a helical paths in the quasi-uniform magnetic field of the tracking requiring a total of five parameters to determine the trajectory. These five parameters are extracted using two or three hits in the inner region of the tracker. The seeds are constructed in the inner part (and then tracks constructed outwards, and not in the opposite manner) because the high granularity of pixel detectors (in contrast to outer strip layers)  ensure low fraction of channels that are hit. Also, particles like pions and electrons interact inelastically with tracker material or lose energy due to bremsstrahlung radiation as they traverse through the tracker to its outer regions making the idea of construcing seeds in the inner region a better choice.

The second step in track generation is track finding which is closely based on the Kalman filter. It extrapolates the seed trajectories along the expected path of a charged particle, beginning with an estimate of the track parameters provided by the trajectory seeds generated in the last step. It then uses the location and uncertainty of detected hits, and estimations of effects such as Coulomb scattering, at successive detector layers, to build track candidates, updating the parameters at each layer. First, using the parameters of the track candidate, evaluated at the current layer, an analytical extrapolation is done that determines which adjacent layers of the detector the trajectory can intersect. This takes into account the current uncertainty in that trajectory just like a Kalman filter. Secondly, a search is perfomed for silicon modules in these layers that are compatible with the extrapolated trajectory. All compatible modules in each layer are then grouped into mutually exclusive groups, such that no two modules in each group overlap. The collection of all hits from from one such module group forms a group of hits. Finally, new track candidates are formed by adding exactly one of the compatible hits from each group, to each original track candidate. The modules in a given group are mutually exclusive and a contribution of more than one hit from each group is not expected. The trajectory parameters of the new candidates are then updated by combining the information from the added hits with the extrapolated trajectory of the original track candidates. Fig~\ref{fig:trackrecon} illustrates the reconstruction efficiency of tracks in case of isolated muons.

The third step in track generation track fitting. In this step the collection of hits from the last step are refitted using a Kalman filter and smoother, to provide a best possible estimate of parameters for each track trajectory. The procedure described above, in conditions as challenging as the LHC, can yields several fake tracks that are not associated with any charged particle passing through the tracker. The fourth and final step applies several quality requirements to set of reconstructed tracks and substantially reduces the fake contribution. The requirements are based on criteria such as the minimum number of layer the track has hits in, how compatible its origin is with a primary vertex, how good a fit they yield etc.

\begin{figure*}
\begin{center}
\includegraphics[width=0.9\textwidth,keepaspectratio]{plots_and_figures/chapter4/trackrecon.png}
\caption{Track reconstruction efficencies for single isolated muons as a function of $\eta$ and \pt~\cite{track_reconstruction}.}
\label{fig:trackrecon}
\end{center}
\end{figure*}


Proton-proton interaction vertices are reconstructed by selecting tracks that are produced promptly in the primary interaction region. The selected tracks are then clustered on the basis of their z-coordinatesat their point of closest approach to the centre of the beam spot, which represents a 3-D profile of the region where the LHC beams collide inside the CMS detector. The exact positions of the vertices are then obtained from these clustered candidates, by using a fitting procedure, called the adaptive vertex fitter~\cite{vertex_fitting}. The vertex which has the largest sum of squared transverse momenta of tracks originating from it is considered the primary interaction vertex. 





\subsection{Electron Reconstruction}
\label{e_recon}
\subsection{Muon Reconstruction}
\label{mu_recon}
\subsection{Jet Reconstruction}
\label{jet_recon}
\subsection{MET, MT and Collinear Mass}
\label{col_mass}
\subsection{Tau Lepton and others}
\label{tau_recon}


\section{Datasets}
\label{datasets}

The data analysed in this search was gathered by the CMS detector in 2016 during proton-proton collisions at the LHC, corresponding to an integrated luminosity of $35.9 fb^{-1}$. This data corresponds to a center-of-mass energy of 13 TeV and a spacing of 25ns between bunch crossings in the LHC with an average of about 30 collisions per bunch crossing. The subset of samples used among all collected by CMS are the ones having at least one isolated muon having transverse energy over 24 GeV, as triggered by the CMS high level isolated muon trigger (HLT\_IsoMu24 in CMS parlance).




% % uncomment the following lines,
% if using chapter-wise bibliography
%
% \bibliographystyle{ndnatbib}
% \bibliography{example}

% % uncomment the following lines,
% if using chapter-wise bibliography
%
% \bibliographystyle{ndnatbib}
% \bibliography{example}




%
%
% Chapter 6
%

\chapter{Event Selection}
\label{event_selection}


\section{H125 Analysis}
\label{h125_evt_selec}

\subsection{Backgrounds}
\label{h125_evt_sel_bkg}

\subsection{Cut Based Selection}
\label{h125_cb_sel}

\subsection{BDT Based Selection}
\label{h125_bdt_Sel}

\subsubsection{Boosted Decision Trees}
\label{bdts}


\section{Heavy Higgs Analysis}
\label{hh_evt_selec}

\subsection{Backgrounds}
\label{hh_evt_sel_bkg}









% % uncomment the following lines,
% if using chapter-wise bibliography
%
% \bibliographystyle{ndnatbib}
% \bibliography{example}




%\chapter{Background Estimation and Validation}
\label{bg_val}
\section{Introduction}
This chapter describes the techniques used for estimation of the backgrounds in the analyses. Each background is estimated individually. For large backgrounds, the estimation is validated using regions enriched in those backgrounds.

\section{h125: \hmue backgrounds}
\label{h125_bg_val}

\subsection{\ztt}
\label{h125_ztt}
The \ztt background is the dominant background in 0-jet and 1-jet categories of the analysis. It is an irreducible background and arises when one $\Pgt$ coming from the $\PZ$ boson decay further decays into a $\Pgm$ and the other decays into a $\Pe$. This background is estimated from simulated monte-carlo events. In a $\PZ \to \ell \ell$ events from Drell-Yan production including \ztt, the $m_{\ell\ell}$ and $\PZ_{\pt}$ distributions are found to be different in data and simulation. In order to correct for this, a set of reweighting factors is calculated using a dedicated control region enriched in $\PZ \to \Pgm\Pgm$ events. The set of reweighting factors are applied as a function of generator-level $m_{\ell\ell}$ and $\PZ_{\pt}$ in the signal region of the analysis. A more detailed study of this effect and calculation of the reweighting factors can be found in the following references~\cite{CMS-PAS-HIG-16-043}.

To validate this estimation, we look at agreement between observed data and simulation in a region enriched in \ztt events. This region is constructed by requiring, in addition to the baseline selection,  the $\pt$ of the $\Pgm<40$ \GeV. The $\pt$ in \ztt events is on softer side of the spectrum compared to other backgrounds which are more spread out, as seen in Fig.~\ref{fig:h125_presel1} (top right). The $M_T(\Pgm)$, as seen in Fig.~\ref{fig:h125_presel1} (bottom right), is required to be less than 60\GeV following similar reasoning. Further the invariant mass of the $\Pe$ and $\Pgm$ is required to be in between 30\GeV and 70\GeV in order to isolate the $\PZ$ peak. The distributions of BDT response and \mcol in this \ztt enriched region are shown in Fig.~\ref{fig:ztt_cr}, for the categories where this background is dominant. The plots show good agreement between data and background.



\begin{figure*}[!htpb]\centering
 \includegraphics[width=0.49\textwidth]{plots_and_figures/chapter6/ztt_cr/0_preselection_BDT_value.pdf}
 \includegraphics[width=0.49\textwidth]{plots_and_figures/chapter6/ztt_cr/1_preselection_BDT_value.pdf} \\
 \includegraphics[width=0.49\textwidth]{plots_and_figures/chapter6/ztt_cr/0_preselection_h_collmass_pfmet.pdf} 
 \includegraphics[width=0.49\textwidth]{plots_and_figures/chapter6/ztt_cr/1_preselection_h_collmass_pfmet.pdf} \\

 \caption{Distributions of BDT response (top) an \mcol (bottom) in \ztt enriched region for 0-jet (left)  and 1-jet (right) categories.}
 \label{fig:ztt_cr}
\end{figure*}


\subsection{\ttb}
\label{h125_ttb}
Tops decay into $\PW$ bosons and a b-quark more than 90\% of the time. The $\PW$ boson can decay leptonically into a $\Pgm$ and $\Pe$ making it a background for the analysis. The b-tagging veto applied at the baseline selection level is able to somewhat supress this background. However it still forms a large fraction of the background for the analysis. In fact, it is the largest background in both 2-jet categories. It is also large in the 1-jet category. We estimate the \ttb background using simulation. The background estimation is validated in two separate control regions enriched in \ttb. The first control region is formed requiring the baseline selection but with a inverted b-tagging veto. In other words, at least 1 b-tagged jet is required to be present in the event. The distributions of BDT response (top) and \mcol (bottom) in this region are shown in Fig.~\ref{fig:tt_cr} for categories where the \ttb background is large. The second control region is constructed using kinematic selection criteria. In particular, in addition to the baseline selection criteria with the b-tag veto removed, we require $M_T(\Pe)$ (see Fig.~\ref{fig:h125_presel2} top left) to be greater than 50\GeV. The distributions of BDT response (top) and \mcol (bottom) in this second control region are shown in Fig.~\ref{fig:tt_cr_cutbased}. Given that the uncertainty bands in these control region plots only contain uncertainties on normalization (and not shape-based uncertainties, as discussed in section~\ref{sig_ext}, and included in the max likelihood fit used to extract results), the data over background estimation  ratio is reasonable in these regions. Further, a normalization uncertainty of 10\% is applied on the \ttb estimation in the signal region based on these control regions.


\begin{figure}[!htpb]\centering
 \includegraphics[width=0.30\textwidth]{plots_and_figures/chapter6/tt_cr/1_preselection_BDT_value.pdf}
 \includegraphics[width=0.30\textwidth]{plots_and_figures/chapter6/tt_cr/21_preselection_BDT_value.pdf} 
 \includegraphics[width=0.30\textwidth]{plots_and_figures/chapter6/tt_cr/22_preselection_BDT_value.pdf} \\
 \includegraphics[width=0.30\textwidth]{plots_and_figures/chapter6/tt_cr/1_preselection_h_collmass_pfmet.pdf}
 \includegraphics[width=0.30\textwidth]{plots_and_figures/chapter6/tt_cr/21_preselection_h_collmass_pfmet.pdf} 
 \includegraphics[width=0.30\textwidth]{plots_and_figures/chapter6/tt_cr/22_preselection_h_collmass_pfmet.pdf} \\


 \caption{Distributions of BDT response (top) an \mcol (bottom) in the first \ttb enriched region, as described in the text.}
 \label{fig:tt_cr}
\end{figure}

\begin{figure}[!htpb]\centering
 \includegraphics[width=0.30\textwidth]{plots_and_figures/chapter6/tt_cr_cutbased/1_preselection_BDT_value.pdf}
 \includegraphics[width=0.30\textwidth]{plots_and_figures/chapter6/tt_cr_cutbased/21_preselection_BDT_value.pdf} 
 \includegraphics[width=0.30\textwidth]{plots_and_figures/chapter6/tt_cr_cutbased/22_preselection_BDT_value.pdf} \\
 \includegraphics[width=0.30\textwidth]{plots_and_figures/chapter6/tt_cr_cutbased/1_preselection_h_collmass_pfmet.pdf}
 \includegraphics[width=0.30\textwidth]{plots_and_figures/chapter6/tt_cr_cutbased/21_preselection_h_collmass_pfmet.pdf} 
 \includegraphics[width=0.30\textwidth]{plots_and_figures/chapter6/tt_cr_cutbased/22_preselection_h_collmass_pfmet.pdf} \\


 \caption{Distributions of BDT response (top) an \mcol (bottom) in the second \ttb enriched region, as described in the text.}
 \label{fig:tt_cr_cutbased}
\end{figure}

\subsection{Misidentified lepton background}
\label{h125_misid_bg}
Another source of background which is relatively much smaller than \ttb or \ztt arises from jets misidentified as leptons in \wjets or SM events comprised uniquely of jets produced through the strong interaction, referred to as quantum chromodynamics (QCD) multijet events. In \wjets events, one lepton candidate is a real lepton from the $\PW$ boson decay while the other lepton is a misidentified jet. In QCD events, both leptons in the final state are misidentified jets. The baseline selection criteria requires the leptons to be well identified and isolated. This makes it difficult for a jet to masquerade as a lepton. In case of the $\Pgm$, this is even more so since it is required to satisfy high $\pt$ thresholds as well. Consequently, these events form a small part of the background. This is in contrast to a final state where the non-prompt lepton is a hadronically decaying $\Pgt$ instead of an electronically decaying one. This background would be much larger in such a case.

The W + jets background contribution to the misidentified-lepton background is estimated using simulation. The QCD multijet contribution is estimated from collision data events where the leptons have like-sign charge. The expected yield from non-QCD processes in this region is subtracted using simulation. The resulting sample is then rescaled to account for the differences between the composition in the like- and opposite-sign charge regions. The scaling factors are extracted from samples enriched  QCD multijet events, and the procedure is illustrated in Ref.~\cite{}. This background is validated in a control region that is obtained by requiring the baseline selection but inverting the isolation criteria. In other words events with well-isolated $\Pgm$ and $\Pe$ are rejected. The particular isolation thresholds required for this region are:$0.1 < I_\text{rel}^{\Pe} < 1$ or $0.15 < I_\text{rel}^{\Pgm} < 0.25$. The distributions of BDT response and \mcol in this qcd enriched region are shown in Fig.~\ref{fig:qcd_cr}. The plots show good agreement between data and background.



\begin{figure*}[!htpb]\centering
 \includegraphics[width=0.49\textwidth]{plots_and_figures/chapter6/qcd_cr/0_preselection_BDT_value.pdf}
 \includegraphics[width=0.49\textwidth]{plots_and_figures/chapter6/qcd_cr/1_preselection_BDT_value.pdf} \\
 \includegraphics[width=0.49\textwidth]{plots_and_figures/chapter6/qcd_cr/0_preselection_h_collmass_pfmet.pdf} 
 \includegraphics[width=0.49\textwidth]{plots_and_figures/chapter6/qcd_cr/1_preselection_h_collmass_pfmet.pdf} \\

 \caption{Distributions of BDT response (top) an \mcol (bottom) in QCD enriched region for 0-jet (left)  and 1-jet (right) categories.}
 \label{fig:qcd_cr}
\end{figure*}

\subsection{Other backgrounds}
\label{h125_other_bg}
The other backgrounds in the analysis make relatively much smaller contrbutions. Electroweak diboson production ($\PW\PW$, $\PW\PZ$ and $\PZ\PZ$) contributes a similar number of events as the misidentified lepton background, and is estimated from simulation. $\PW\PW$ events make the largest contribution, followed by $\PW\PZ$ and $\PZ\PZ$ events. This is because $\PW\PZ$ and $\PZ\PZ$ events  have additional leptons in their final state which have to miss detection in order for the event to be a background. SM decays of the h boson also forms a small but non-negligible background. These come particularly from $\text{h} \to \Pgt\Pgt$  and $\text{h} \to \PW\PW$ decays. Other backgrounds include $Z\to\ell\ell$ $(\ell = \Pe, \Pgm)+\text{jets}$, single-top quark production and $\PW\gamma^{(*)}+\text{jets}$. All of these are estimated using simulation.  

\section{Heavy Higgs: \Hmue backgrounds }
\label{H_bg_val}
The background processes in the \Hmue analysis are  similar  to \hmue but differ in relative contribution, and are overall much smaller. This is due to the fact that the \Hmue analyses searches for LFV decay in a higher mass, higher \pt region. In particular, \ztt background which is the most dominant in \hmue is now very small. The \ztt background peaks around the $\PZ$ boson mass, and the  high $\pt$ cuts in this anlysis reject most of these events. The dominant backgrounds in \Hmue are \ttb production, followed by electroweak diboson production which have a relatively flatter $\pt$ distribution and survive the strict $\pt$ requirements.

\ttb production is the largest background in the \Hmue analysis. We estimate this background using simulation. A control region enriched in \ttb events is constructed by requiring the baseline selection with the b-tag veto removed, and with the additional requirement that at least 1 b-tagged jet be present. Fig.~{\ref{fig:tt_cr_nosf} (left) shows the \mcol distribution of this sample. To take into account the residual data to background estimation difference, an overall normalization scale factor of 0.886 is extracted from this region, and is applied to the background estimation in the signal region. The same control region above is shown in Fig.~{\ref{fig:tt_cr_nosf} (right), after the background has been scaled by the above factor for illustration. Distributions of several other kinematic variables (after the above rescaling) in the \ttb control region are shown in Fig.~\ref{fig:tt_cr}. They show reasonable agreement between data and estimated background.

\begin{figure*}[htpb]
  \begin{center}

    \includegraphics[width=0.48\textwidth]{plots_and_figures/chapter6/tt_cr_hm/ttbarNoSF.pdf}
    \includegraphics[width=0.48\textwidth]{plots_and_figures/chapter6/tt_cr_hm/ttbarSF.pdf}
  \end{center}
  \caption{\mcol distribution in \ttb enriched control region as defined in the text before the application of the scale factor (left) and after (right),for the \Hmue analysis.}
  \label{fig:tt_cr_nosf}
\end{figure*}

\begin{figure*}[htbp]
     \centering
      \includegraphics[width=0.48\textwidth]{plots_and_figures/chapter6/tt_cr_hm/log_mutaue_1jet_presel_mPt.pdf}
      \includegraphics[width=0.48\textwidth]{plots_and_figures/chapter6/tt_cr_hm/mutaue_1jet_presel_dphiemu.pdf}\\
      \includegraphics[width=0.48\textwidth]{plots_and_figures/chapter6/tt_cr_hm/mutaue_1jet_presel_dphiEMet.pdf}
      \includegraphics[width=0.48\textwidth]{plots_and_figures/chapter6/tt_cr_hm/mutaue_1jet_presel_dphiMuMet.pdf}\\
     \caption{Distributions of several kinematic variables in the \ttb enriched control region for \Hmue analysis.}
     \label{fig:tt_cr}
\end{figure*}

Electroweak diboson production ($\PW\PW$, $\PW\PZ$ and $\PZ\PZ$) forms the next largest background in \Hmue analysis. It is estimated using simulation. All other backgrounds are much smaller. This can be seen from the distributions of kinematic variables after baseline selection, as can be seen from Figs.~\ref{fig:Hmutaue_presel1} and ~\ref{fig:Hmutaue_presel2}. The misidentified lepton background is even smaller here than \hmue. The higher $\pt$ requirement makes it even less likely for jets to be able to be misidentified as leptons. This background is estimated using the same technique as \hmue, as described in section~\ref{h125_misid_bg}. The $Z\to\ell\ell$ $(\ell = \Pe, \Pgm)+\text{jets}$ and \ztt backgrounds are estimated from simulation. Other backgrounds include SM h boson decays, $\text{h} \to \PW\PW$, $\text{h} \to \Pgt\Pgt$, single-top quark production and $\PW\gamma^{(*)}+\text{jets}$, and are also estimated using simulation. 

 
    




% % uncomment the following lines,
% if using chapter-wise bibliography
%
% \bibliographystyle{ndnatbib}
% \bibliography{example}





%
% Chapter 7
%

\chapter{Signal extraction and systematic uncertainties}
\label{sig_ext}


\section{H125 Analysis}
\label{h125_sys}

\subsection{Theoretical uncertainties}
\label{theo_uncert}

\subsection{Experminetal uncertainties}
\label{exp_uncert}

\subsection{Signal extraction}
\label{sig_ext}


\section{Heavy Higgs Analysis}
\label{hh_sys}

\subsection{Theoretical uncertainties}
\label{theo_uncert}

\subsection{Experminetal uncertainties}
\label{exp_uncert}

\subsection{Signal extraction}
\label{sig_ext}

% % uncomment the following lines,
% if using chapter-wise bibliography
%
% \bibliographystyle{ndnatbib}
% \bibliography{example}





%
% Chapter 8
%

\chapter{Results}
\epigraph{There are two possible outcomes: if the result confirms the hypothesis, then you've made a measurement. If the result is contrary to the hypothesis, then you've made a discovery.}{\textit{Enrico Fermi}}
\label{results}
In this chapter the results of both the searches are presented. The results for the $\hmue$ search are first presented. Results for the $\Hmue$ search follow.

\subsection{$\hmue$ results}
The resulting distributions of the signal variable (after applying all selection requirements as outlined in ~\ref{h125_evt_sel}) are fit using a binned maximum likelihood fit. The entire procedure is described in detail in ~\ref{stat_meth}. All systematic uncertainties are included as nuisance parameters, and the fit is performed simultaneously across all categories. The BDT response distributions of signal and background are shown superimposed for each category in Fig~\ref{fig:BDT_dist_hmue}. The distribution of $\mcol$ for the $\mcol$-fit analysis are also shown in Fig~\ref{fig:mcol_dist_hmue}. We do not observe an excess of signal over expected background. Hence, upper exclusion limits on $\mathcal{B}(\hmue)$ are set, following the procedure described in ~\ref{exc_cal}. In table~\ref{table:hmue_limits}, the median expected limits, observed limits and the best fit branching fractions for $\mathcal{B}(\hmue)$ are summarized. As noted earlier in this thesis, the tau lepton coming from the Higgs can also decay hadronically. This channel of the LFV Higgs decay, i.e. $\hmuhad$ is studied in an analyses by different members of the same research team~\cite{HIG-17-001}. The limits on $\mathcal{B}(\hmuhad)$ from that search are combined with limits on $\mathcal{B}(\hmue)$, as calculated above from the search described here. All limits are summarized graphically in Figure~\ref{fig:hmue_limits_brazil}. The combined observed (median expected) upper limits on $\mathcal{B}(\hmu)$ is 0.25 (0.25)\,\% at 95\% CL, for the BDT-fit analysis. The combined best fit branching fraction of $\mathcal{B}(\hmu)$ is found to be $0.00 \pm 0.12$, also for the BDT-fit analysis. It is important to note that the 2.4\,$\sigma$ excess observed by the earlier CMS search with 8 TeV data~\cite{Khachatryan:2015kon} has now been excluded by this search.


\begin{table*}[!htpb]
 \begin{center}
   \caption{Expected and observed upper limits at 95\% CL, and best fit branching fractions in percent for each individual jet category, and combined, for the $\hmue$ analysis.}
   \scalebox{0.85}{
\begin{tabular}{*{6}{c}}
\multicolumn{6}{c}{Expected limits~(\%) } \\ \hline
                       &  \multicolumn{1}{c}{0-jet}   & \multicolumn{1}{c}{1-jet}    &  \multicolumn{1}{c}{2-jets} & \multicolumn{1}{c}{VBF}  & \multicolumn{1}{c}{Combined}                 \\  \cline{2-6}
BDT fit analysis 	 &  $<$0.83   	 &  $<$1.19   	 &  $<$1.98   	 &  $<$1.62   	 &  $<$0.59    \\
$\mcol$ fit analysis 	 &  $<$1.01   	 &  $<$1.47   	 &  $<$3.23   	 &  $<$1.73   	 &  $<$0.75    \\
\cline{2-6}
\\ [\cmsTabSkip]
\multicolumn{6}{c}{Observed limits~(\%)} \\ \hline
                       &  \multicolumn{1}{c}{0-jet}   & \multicolumn{1}{c}{1-jet}    &  \multicolumn{1}{c}{2-jets} & \multicolumn{1}{c}{VBF} &\multicolumn{1}{c}{Combined}                 \\ \cline{2-6}
BDT fit analysis   		 & $<$1.30   	 & $<$1.34   	 & $<$2.27   	 & $<$1.79   	 & $<$0.86    \\
$\mcol$ fit analysis   		 & $<$1.08   	 & $<$1.35   	 & $<$3.33   	 & $<$1.40   	 & $<$0.71    \\
\cline{2-6}
\\[\cmsTabSkip]
\multicolumn{6}{c}{Best fit branching fractions~(\%)} \\ \hline
                       &  \multicolumn{1}{c}{0-jet}   & \multicolumn{1}{c}{1-jet}    &  \multicolumn{1}{c}{2-jets} & \multicolumn{1}{c}{VBF} &\multicolumn{1}{c}{Combined}                 \\  \cline{2-6}
BDT fit analysis    		 & 0.61 $\pm$ 0.36  	 & 0.22 $\pm$ 0.46  	 & 0.39 $\pm$ 0.83  	 & 0.10 $\pm$ 1.37  	 & 0.35 $\pm$ 0.26  \\
$\mcol$ fit analysis    		 & 0.13 $\pm$ 0.43  	 & $-0.22$ $\pm$ 0.75  	 & 0.22 $\pm$ 1.39  	 & $-1.73$ $\pm$ 1.05  	 & $-0.04$ $\pm$ 0.33  \\
\cline{2-6}
combined $\Pgm\Pgt$ (BDT fit)  & \multicolumn{5}{c}{ 0.00 $\pm$ 0.12 } \\ \hline
\end{tabular}}
\end{center}
\label{table:hmue_limits}
\end{table*}

\begin{figure*}[!htpb]\centering
 \includegraphics[width=0.49\textwidth]{plots_and_figures/chapter8/h125/0jetBDT.pdf}
 \includegraphics[width=0.49\textwidth]{plots_and_figures/chapter8/h125/1jetBDT.pdf} \\
 \includegraphics[width=0.49\textwidth]{plots_and_figures/chapter8/h125/2jetggBDT.pdf}
 \includegraphics[width=0.49\textwidth]{plots_and_figures/chapter8/h125/2jetvbBDT.pdf} 
\caption{Distribution of BDT response in each category comparing signal and background estimations to observed collision data, for $\hmue$ analysis. The bottom panel show the ratio of observed data and fitted background in each bin~\cite{HIG-17-001}}
 \label{fig:BDT_dist_hmue}
\end{figure*}

\begin{figure*}[!htpb]\centering
 \includegraphics[width=0.49\textwidth]{plots_and_figures/chapter8/h125/0jetmcol.pdf}
 \includegraphics[width=0.49\textwidth]{plots_and_figures/chapter8/h125/1jetmcol.pdf} \\
 \includegraphics[width=0.49\textwidth]{plots_and_figures/chapter8/h125/2jetggmcol.pdf}
 \includegraphics[width=0.49\textwidth]{plots_and_figures/chapter8/h125/2jetvbmcol.pdf} 
\caption{Distribution of $\mcol$ response in each category comparing signal and background estimations to observed collision data, for $\hmue$ analysis. The bottom panel show the ratio of observed data and fitted background in each bin~\cite{HIG-17-001}}
 \label{fig:mcol_dist_hmue}
\end{figure*}

\begin{figure*}[!htpb]\centering
 \includegraphics[width=0.49\textwidth]{plots_and_figures/chapter8/h125/brazilflagBDT.pdf}
 \includegraphics[width=0.49\textwidth]{plots_and_figures/chapter8/h125/brazilflagmcol.pdf} \\
 \caption{Observed and median expected upper exclusion limits for $\hmue$, $\hmuhad$ and combined $\hmu$ channels, for the BDT fit (left) and $\mcol$ fit analysis (right). The $\pm 1 \sigma$ and $\pm 2 \sigma$ bands for expected limits are also shown in light green and yellow respectively~\cite{HIG-17-001}.}
 \label{fig:hmue_limits_brazil}
\end{figure*}


The constraints on $\mathcal{B}(\hmu)$ can be transformed into constraints on Lepton Flavor Violating Yukawa Couplings ($Y_{\Pgm\Pgt},Y_{\Pgt\Pgm}$). These couplings represent the strength of an interaction and are related to the decay width $\Gamma(\hmu)$ in the following way~\cite{Harnik:2012pb}:
\begin{equation}                                                                                                                                                                                                 
\Gamma(\hmu)=\frac{m_{h}}{8\pi}(|Y_{\Pgm\Pgt}|^2 + |Y_{\Pgt\Pgm}|^2).                                                          
\label{eq:yuk1}
\end{equation}

The decay width is also related to the branching fraction, $\mathcal{B}(\hmu)$ according to the following equation:
\begin{equation}                                                                                                                                                                                                \mathcal{B}(\hmu)=\frac{\Gamma(\hmu)}{\Gamma(\hmu) + \Gamma_{SM}}.
\label{eq:yuk2}
\end{equation}

,where the SM Higgs decay width is assumed to be $\Gamma_{SM}=4.1$ MeV~\cite{Denner:2011mq} for $m_{\PH}=125$\GeV. Using equations~\ref{eq:yuk1} and ~\ref{eq:yuk2}, we derive the constraints on Yukawa couplings at 95\% CL. The limits for the Yukawa couplings are summarized in Table~\ref{table:yuk_coup}. Fig.~\ref{fig:yuk_coup} pictorially summarizes all existing limits on Yukawa couplings from different direct and indirect searches. It also shows the theoretical "naturalness" limit considering/expecting LFV couplings to be smaller than those of couplings for SM decays of the Higgs~\cite{Harnik:2012pb}, which can be considered a benchmark for sensitivity of this search. The limits derived from this search are most stringent till date, and surpass the above benchmark. 

\begin{table*}[!hbtp]
 \centering
  \caption{95\% CL observed upper limit on the Yukawa couplings,  for the BDT fit and the \mcol fit analysis.}
 \label{table:yuk_coup}
\begin{tabular}{|ccc| }
   \hline
                        & BDT fit  &  \mcol fit \\ \hline
$\sqrt{|Y_{\Pgm\Pgt}|^{2}+|Y_{\Pgt\Pgm}|^{2}}$   & $<1.43\times 10^{-3}$ &  $<2.05\times 10^{-3}$  \\
  \hline
\end{tabular}
\end{table*}

\begin{figure*}[!htpb]\centering
 \includegraphics[width=0.6\textwidth]{plots_and_figures/chapter8/h125/yukawa.pdf}
 \caption{Observed (black solid)and median expected (red dashed) upper limits on $\hmu$ Yukawa couplings from this analysis. The light green and yellow bands show the $\pm 1 \sigma$ and $\pm 2 \sigma$ spreads of the expected limit. Blue solid line shows the result from the previous CMS search with 8 TeV data~\cite{Khachatryan:2015kon}. The naturalness limit is shown as a purple straight line.~\cite{HIG-17-001}}
 \label{fig:yuk_coup}
\end{figure*}


\subsection{$\Hmue$ results}
The resulting $\mcol$ distributions for signal and background estimation (after applying all selection requirements as outlined in ~\ref{HH_evt_sel}), after a binned maximum likelihood fit, are shown superimposed along with the observed data Fig~\ref{fig:mcol_dist_Hmue}. All systematic uncertainties are included as nuisance parameters, and the fit is performed simultaneously across all categories. We do not observe an excess over expected background in the entire range. Unlike the $\hmue$ analysis described above where the production cross-section of the SM Higgs boson is known, here we are looking for LFV decay of a hypothetical heavy Higgs bosons of different masses. Hence, we set upper exclusion limits on production cross-section times branching fraction, $\sigma(\textrm{gg}\rightarrow \PH)\times\mathcal{B}(\Hmue)$. The procedure is the same as used above and described in~\ref{exc_cal}. The observed and median expected upper limits at 95\% CL on $\sigma(\textrm{gg}\rightarrow \PH)\times\mathcal{B}(\Hmue)$ are summarized in table~\ref{table:limits_Hmue} for different categories and Higgs masses. The limits are also summarized graphically in Figure~\ref{fig:limits_Hmue}. The observed (median expected) limits range from 159.4 (95.6)\,pb to 2.9 (4.9)\,pb for heavy Higgs masses in the range between 200 and 900\,GeV. This search was combined with LFV heavy Higgs decay search with the tau lepton decaying hadronically, i.e. $\Hmuhad$ to produce constraints $\Hmu$. The combined observed (median expected) upper limits on $\sigma(\textrm{gg}\rightarrow \PH)\times\mathcal{B}(\Hmu)$ range from 51.9 (57.4)\,pb to 1.6 (2.1)\,pb. These limits are shown graphically in Figure~\ref{fig:limits_Hmu}. This is the first direct search till date to set limits on this decay.      

\begin{figure*}[!htpb]\centering
 \includegraphics[width=0.49\textwidth]{plots_and_figures/chapter8/highmass/log_low_me_ch1_HMuTau_mutaue_1_2016_postfit_colmass_postfit.pdf}
 \includegraphics[width=0.49\textwidth]{plots_and_figures/chapter8/highmass/log_low_me_ch1_HMuTau_mutaue_2_2016_postfit_colmass_postfit.pdf} \\
 \includegraphics[width=0.49\textwidth]{plots_and_figures/chapter8/highmass/log_high_me_ch1_HMuTau_mutaue_1_2016_postfit_colmass_postfit.pdf}
 \includegraphics[width=0.49\textwidth]{plots_and_figures/chapter8/highmass/log_high_me_ch1_HMuTau_mutaue_2_2016_postfit_colmass_postfit.pdf} 
\caption{Distribution of $\mcol$ in 0-jet (left) and 1-jet (right) for lowmass (top) and highmass (range), comparing signal and background estimations to observed collision data, for $\Hmue$ analysis. The bottom panel show the ratio of observed data and fitted background in each bin~\cite{HIG-18-017}}
 \label{fig:mcol_dist_Hmue}
\end{figure*}


\begin{figure*}[!htpb]\centering
 \includegraphics[width=0.49\textwidth]{plots_and_figures/chapter8/highmass/Figure_004-c.pdf}
 \includegraphics[width=0.49\textwidth]{plots_and_figures/chapter8/highmass/Figure_004-d.pdf} \\
 \includegraphics[width=0.60\textwidth]{plots_and_figures/chapter8/highmass/Figure_005-b.pdf}
\caption{Observed and Median expected 95\% upper exclusion limits for 0-jet (upper left), 1-jet (upper right) and combined (bottom),for the $\Hmue$ analysis. ~\cite{HIG-18-017}}
 \label{fig:limits_Hmue}
\end{figure*}



\begin{table*}
\caption{The observed (median expected) 95\% CL upper limits on $\sigma(\textrm{gg}\rightarrow \PH) \times\mathcal{B}(\Hmue)$.}
\begin{center}
\begin{tabular}{c|c|c|c}
\hline
$m_{\PH}$ (GeV) & 0 jet & 1 jet  & comb\\
\hline
200 &147.8 (107.5) & 262.1 (209.8)& 159.4 (95.6) \\
300 &30.1 (49.8) & 100.8 (108.6) & 29.3 (45.2) \\
450 &31.1 (17.5) & 35.3 (32.8) )& 23.7 (20.4) \\
600 &8.1 (10.4)& 15.2 (17.9)& 6.8 (8.9) \\
750 &6.5 (8.0)& 7.8 (18.2)& 4.7 (6.1) \\
900 &4.4 (6.9)& 5.6 (15.4)& 2.9 (4.9) \\
\hline
\end{tabular}
\label{table:limits_Hmue}
\end{center}
\end{table*}


\begin{figure*}[!htpb]\centering
 \includegraphics[width=0.8\textwidth]{plots_and_figures/chapter8/highmass/Figure_005-c.pdf}
\caption{Observed and Median expected 95\% upper exclusion limits for combined $\Hmu$ analysis~\cite{HIG-18-017}.}
 \label{fig:limits_Hmu}
\end{figure*}



% % uncomment the following lines,
% if using chapter-wise bibliography
%
% \bibliographystyle{ndnatbib}
% \bibliography{example}


%
% Chapter 9
%

\chapter{Conclusion}
\label{conclusion}

This dissertation describes two searches for Beyond the Standard Model (BSM) decays. Both searches were performed with data collected by the CMS detector in 2016, in proton-proton collisions at the LHC, at a center-of-mass energy of 13\,TeV.

The search for SM Higgs (h) decaying into a muon and an electronically decaying tau ($\hmue$) is now a pubslished result in a peer-reviewed journal~\cite{HIG-17-001}. This search, (in combination with $\hmuhad$) has set most stringent upper bounds till date on the branching fraction of h decaying to $\mu\tau$. The observed (median expected) upper limits on $\mathcal{B}(\hmu)$ is 0.25 (0.25)\,\% at 95\% CL. Upper limit on off-diagonal $\mu\tau$ Yukawa couplings, derived from the above constraint, is also set to be $\sqrt{|Y_{\Pgm\Pgt}|^{2}+|Y_{\Pgt\Pgm}|^{2}}<1.43\times 10^{-3}$ at 95\% CL. These limits constitute a significant improvement over all previous results.

The search for flavour violating decays of a neutral heavy Higgs boson (H) into a muon and an electronically decaying tau ($\hmue$) is complete, and on its way to being a part of a peer-reviewed publication. This search, (in combination with $\Hmuhad$) has set upper limits on product of H cross-section and branching ratio to $\mu\tau$. These observed (median expected) limits on $\sigma(\textrm{gg}\rightarrow \PH)\times\mathcal{B}(\Hmu)$ range from 51.9 (57.4)\,pb to 1.6 (2.1)\,pb for H masses in the range between 200 and 900 GeV. This search is the first direct search to set limits on this decay.

As the LHC excellent performance in delivering proton-proton collisions continues, and CMS collects more an more  data, there is room for above results to above. The very next step would be to perform these searches with the entire dataset collected in the run II (2015-2018) period. This would amount to more than 3 times the data used in the above searches, and with more innovative techniques to select signal events and reduce backgrounds, the sensitivity of the searches can be increased. 







%
% Appendix (optional)
%

\appendix

\chapter{Boosted Decision Trees}
\label{chap_BDT} 

\section{Introduction}
%\include{appendixB}

%
% Back stuff
%

% % comment out the following three lines
% if using chapter-wise bibliography

 \backmatter

  \bibliographystyle{unsrtnat}%\bibliographystyle{abbrvnat} % The standard abbrvnat style should be acceptable. Also provided with both the advanced and standard
  \bibliography{thesis_v0}  % distributions are nddiss2e and nddiss2enoarticletitles style options.
% If you prefer to manually enter your bibliography, that is fine. Comment out the previous two lines, and enter your bibliography
% as usual. Note that if you choose this route, formatting the bibliography is your responsibility. An example is below, including the
% optional arguments necessary for author-date style citations.
%	\begin{thebibliography}{9}
%		\bibitem[Galmira(1998)]{galmira98:_gnus_milit} G.\ Galmira. Gnus and the military -- a secret conspiracy? \emph{Growing Towards Gnu}, III(7):22--183, September 1998.
%		
%		\bibitem[Ganston and Greenfield(1998)]{gnus98:_gerry_ganst} G.\ Ganston and G.\ Greenfield. \emph{Gnus and You: The Art of Being New}. volume I. Grapping Books, NY, August, 1998.
%		
%		\bibitem[Gloonston(1998)]{gloonston98:_gnuly_discov_gnus} G.\ Gloonston. Newly discovered gnus: The LoG. \emph{Growing Towards Gnu}, II(12):23---57, March 1998.
%		
%		\bibitem[Greenfield(1996)]{greenfield96:_gettin_know_gnu} G.\ Greenfield. \emph{Getting to Know Gnu}. PhD thesis, Geoffrey Garfield School of Gnus, August 1996.
%		
%		\bibitem[van Gairley(2000)]{gairley2000} G.\ van Gairley. Gnu's review. Website, 2000. \url{http://www.gairley.gnu}.
%	\end{thebibliography}

\end{document}

% End of ``example.tex''
